% Options for packages loaded elsewhere
\PassOptionsToPackage{unicode}{hyperref}
\PassOptionsToPackage{hyphens}{url}
%
\documentclass[
  11pt,
]{article}
\usepackage{amsmath,amssymb}
\usepackage{iftex}
\ifPDFTeX
  \usepackage[T1]{fontenc}
  \usepackage[utf8]{inputenc}
  \usepackage{textcomp} % provide euro and other symbols
\else % if luatex or xetex
  \usepackage{unicode-math} % this also loads fontspec
  \defaultfontfeatures{Scale=MatchLowercase}
  \defaultfontfeatures[\rmfamily]{Ligatures=TeX,Scale=1}
\fi
\usepackage{lmodern}
\ifPDFTeX\else
  % xetex/luatex font selection
\fi
% Use upquote if available, for straight quotes in verbatim environments
\IfFileExists{upquote.sty}{\usepackage{upquote}}{}
\IfFileExists{microtype.sty}{% use microtype if available
  \usepackage[]{microtype}
  \UseMicrotypeSet[protrusion]{basicmath} % disable protrusion for tt fonts
}{}
\makeatletter
\@ifundefined{KOMAClassName}{% if non-KOMA class
  \IfFileExists{parskip.sty}{%
    \usepackage{parskip}
  }{% else
    \setlength{\parindent}{0pt}
    \setlength{\parskip}{6pt plus 2pt minus 1pt}}
}{% if KOMA class
  \KOMAoptions{parskip=half}}
\makeatother
\usepackage{xcolor}
\usepackage[margin=1in]{geometry}
\usepackage{color}
\usepackage{fancyvrb}
\newcommand{\VerbBar}{|}
\newcommand{\VERB}{\Verb[commandchars=\\\{\}]}
\DefineVerbatimEnvironment{Highlighting}{Verbatim}{commandchars=\\\{\}}
% Add ',fontsize=\small' for more characters per line
\usepackage{framed}
\definecolor{shadecolor}{RGB}{248,248,248}
\newenvironment{Shaded}{\begin{snugshade}}{\end{snugshade}}
\newcommand{\AlertTok}[1]{\textcolor[rgb]{0.94,0.16,0.16}{#1}}
\newcommand{\AnnotationTok}[1]{\textcolor[rgb]{0.56,0.35,0.01}{\textbf{\textit{#1}}}}
\newcommand{\AttributeTok}[1]{\textcolor[rgb]{0.13,0.29,0.53}{#1}}
\newcommand{\BaseNTok}[1]{\textcolor[rgb]{0.00,0.00,0.81}{#1}}
\newcommand{\BuiltInTok}[1]{#1}
\newcommand{\CharTok}[1]{\textcolor[rgb]{0.31,0.60,0.02}{#1}}
\newcommand{\CommentTok}[1]{\textcolor[rgb]{0.56,0.35,0.01}{\textit{#1}}}
\newcommand{\CommentVarTok}[1]{\textcolor[rgb]{0.56,0.35,0.01}{\textbf{\textit{#1}}}}
\newcommand{\ConstantTok}[1]{\textcolor[rgb]{0.56,0.35,0.01}{#1}}
\newcommand{\ControlFlowTok}[1]{\textcolor[rgb]{0.13,0.29,0.53}{\textbf{#1}}}
\newcommand{\DataTypeTok}[1]{\textcolor[rgb]{0.13,0.29,0.53}{#1}}
\newcommand{\DecValTok}[1]{\textcolor[rgb]{0.00,0.00,0.81}{#1}}
\newcommand{\DocumentationTok}[1]{\textcolor[rgb]{0.56,0.35,0.01}{\textbf{\textit{#1}}}}
\newcommand{\ErrorTok}[1]{\textcolor[rgb]{0.64,0.00,0.00}{\textbf{#1}}}
\newcommand{\ExtensionTok}[1]{#1}
\newcommand{\FloatTok}[1]{\textcolor[rgb]{0.00,0.00,0.81}{#1}}
\newcommand{\FunctionTok}[1]{\textcolor[rgb]{0.13,0.29,0.53}{\textbf{#1}}}
\newcommand{\ImportTok}[1]{#1}
\newcommand{\InformationTok}[1]{\textcolor[rgb]{0.56,0.35,0.01}{\textbf{\textit{#1}}}}
\newcommand{\KeywordTok}[1]{\textcolor[rgb]{0.13,0.29,0.53}{\textbf{#1}}}
\newcommand{\NormalTok}[1]{#1}
\newcommand{\OperatorTok}[1]{\textcolor[rgb]{0.81,0.36,0.00}{\textbf{#1}}}
\newcommand{\OtherTok}[1]{\textcolor[rgb]{0.56,0.35,0.01}{#1}}
\newcommand{\PreprocessorTok}[1]{\textcolor[rgb]{0.56,0.35,0.01}{\textit{#1}}}
\newcommand{\RegionMarkerTok}[1]{#1}
\newcommand{\SpecialCharTok}[1]{\textcolor[rgb]{0.81,0.36,0.00}{\textbf{#1}}}
\newcommand{\SpecialStringTok}[1]{\textcolor[rgb]{0.31,0.60,0.02}{#1}}
\newcommand{\StringTok}[1]{\textcolor[rgb]{0.31,0.60,0.02}{#1}}
\newcommand{\VariableTok}[1]{\textcolor[rgb]{0.00,0.00,0.00}{#1}}
\newcommand{\VerbatimStringTok}[1]{\textcolor[rgb]{0.31,0.60,0.02}{#1}}
\newcommand{\WarningTok}[1]{\textcolor[rgb]{0.56,0.35,0.01}{\textbf{\textit{#1}}}}
\usepackage{graphicx}
\makeatletter
\def\maxwidth{\ifdim\Gin@nat@width>\linewidth\linewidth\else\Gin@nat@width\fi}
\def\maxheight{\ifdim\Gin@nat@height>\textheight\textheight\else\Gin@nat@height\fi}
\makeatother
% Scale images if necessary, so that they will not overflow the page
% margins by default, and it is still possible to overwrite the defaults
% using explicit options in \includegraphics[width, height, ...]{}
\setkeys{Gin}{width=\maxwidth,height=\maxheight,keepaspectratio}
% Set default figure placement to htbp
\makeatletter
\def\fps@figure{htbp}
\makeatother
\setlength{\emergencystretch}{3em} % prevent overfull lines
\providecommand{\tightlist}{%
  \setlength{\itemsep}{0pt}\setlength{\parskip}{0pt}}
\setcounter{secnumdepth}{-\maxdimen} % remove section numbering
\ifLuaTeX
  \usepackage{selnolig}  % disable illegal ligatures
\fi
\IfFileExists{bookmark.sty}{\usepackage{bookmark}}{\usepackage{hyperref}}
\IfFileExists{xurl.sty}{\usepackage{xurl}}{} % add URL line breaks if available
\urlstyle{same}
\hypersetup{
  pdftitle={Final Project},
  pdfauthor={Group 1:Boyeon Kim, Donguk Yoom, JungYoon Choi, Semin Seo, Hanseung Jang, Sanghyun Lee, Helena Burke, Seokyeong Park, Dahye Chung},
  hidelinks,
  pdfcreator={LaTeX via pandoc}}

\title{Final Project}
\author{Group 1:Boyeon Kim, Donguk Yoom, JungYoon Choi, Semin Seo,
Hanseung Jang, Sanghyun Lee, Helena Burke, Seokyeong Park, Dahye Chung}
\date{2023-07-17}

\begin{document}
\maketitle

\begin{Shaded}
\begin{Highlighting}[]
\FunctionTok{library}\NormalTok{(tidyr)}
\FunctionTok{library}\NormalTok{(ggplot2)}
\FunctionTok{library}\NormalTok{(ggmosaic)}
\FunctionTok{library}\NormalTok{(dplyr)}
\FunctionTok{library}\NormalTok{(readr)}
\FunctionTok{library}\NormalTok{(class)}
\end{Highlighting}
\end{Shaded}

\begin{Shaded}
\begin{Highlighting}[]
\FunctionTok{library}\NormalTok{(tidyr)}
\FunctionTok{library}\NormalTok{(ggplot2)}
\FunctionTok{library}\NormalTok{(dplyr)}

\CommentTok{\# 데이터 가져오기}
\NormalTok{Sleep\_health\_and\_lifestyle\_dataset }\OtherTok{\textless{}{-}} \FunctionTok{read\_csv}\NormalTok{(}\AttributeTok{file =} \StringTok{"Sleep\_health\_and\_lifestyle\_dataset.csv"}\NormalTok{,}
  \AttributeTok{col\_types =} \FunctionTok{cols}\NormalTok{(}
    \StringTok{\textquotesingle{}Sleep Duration\textquotesingle{}} \OtherTok{=} \FunctionTok{col\_character}\NormalTok{(),}
    \StringTok{\textquotesingle{}Stress Level\textquotesingle{}} \OtherTok{=} \FunctionTok{col\_character}\NormalTok{(),}
    \StringTok{\textquotesingle{}Physical Activity Level\textquotesingle{}} \OtherTok{=} \FunctionTok{col\_character}\NormalTok{(),}
    \StringTok{\textquotesingle{}Quality of Sleep\textquotesingle{}} \OtherTok{=} \FunctionTok{col\_character}\NormalTok{(),}
    \StringTok{\textquotesingle{}BMI Category\textquotesingle{}} \OtherTok{=} \FunctionTok{col\_character}\NormalTok{(),}
    \StringTok{\textquotesingle{}Blood Pressure\textquotesingle{}} \OtherTok{=} \FunctionTok{col\_character}\NormalTok{(),}
    \StringTok{\textquotesingle{}Heart Rate\textquotesingle{}} \OtherTok{=} \FunctionTok{col\_character}\NormalTok{(),}
    \StringTok{\textquotesingle{}Daily Steps\textquotesingle{}} \OtherTok{=} \FunctionTok{col\_character}\NormalTok{(),}
    \StringTok{\textquotesingle{}Sleep Disorder\textquotesingle{}} \OtherTok{=} \FunctionTok{col\_character}\NormalTok{()}
\NormalTok{  ))}

\CommentTok{\# 데이터 클리닝}
\NormalTok{Sleep\_health\_and\_lifestyle\_dataset\_renamed }\OtherTok{\textless{}{-}}\NormalTok{ Sleep\_health\_and\_lifestyle\_dataset }\SpecialCharTok{\%\textgreater{}\%}
  \FunctionTok{rename}\NormalTok{(}\AttributeTok{ID =} \StringTok{\textquotesingle{}Person ID\textquotesingle{}}\NormalTok{,}
         \AttributeTok{Duration =} \StringTok{\textquotesingle{}Sleep Duration\textquotesingle{}}\NormalTok{,}
         \AttributeTok{Stress =} \StringTok{\textquotesingle{}Stress Level\textquotesingle{}}\NormalTok{,}
         \AttributeTok{Physical =} \StringTok{\textquotesingle{}Physical Activity Level\textquotesingle{}}\NormalTok{,}
         \AttributeTok{Quality =} \StringTok{\textquotesingle{}Quality of Sleep\textquotesingle{}}\NormalTok{,}
         \AttributeTok{BMI =} \StringTok{\textquotesingle{}BMI Category\textquotesingle{}}\NormalTok{,}
         \AttributeTok{BPressure =} \StringTok{\textquotesingle{}Blood Pressure\textquotesingle{}}\NormalTok{,}
         \AttributeTok{HRate =} \StringTok{\textquotesingle{}Heart Rate\textquotesingle{}}\NormalTok{,}
         \AttributeTok{DSteps =} \StringTok{\textquotesingle{}Daily Steps\textquotesingle{}}\NormalTok{,}
         \AttributeTok{Disorder =} \StringTok{\textquotesingle{}Sleep Disorder\textquotesingle{}}\NormalTok{)}
\end{Highlighting}
\end{Shaded}

\begin{Shaded}
\begin{Highlighting}[]
\CommentTok{\# Sufficient Sleep T/F}
\NormalTok{sleep\_data }\OtherTok{\textless{}{-}}\NormalTok{ Sleep\_health\_and\_lifestyle\_dataset\_renamed }\SpecialCharTok{\%\textgreater{}\%}
    \FunctionTok{mutate}\NormalTok{(}\AttributeTok{sufficient\_sleep =} \FunctionTok{as.logical}\NormalTok{(Duration }\SpecialCharTok{\textgreater{}=} \FloatTok{7.0}\NormalTok{))}

\NormalTok{sleep\_data }\SpecialCharTok{\%\textgreater{}\%}
  \FunctionTok{pivot\_longer}\NormalTok{(}\AttributeTok{cols =} \FunctionTok{c}\NormalTok{(Disorder), }\AttributeTok{names\_to =} \StringTok{"variable"}\NormalTok{, }\AttributeTok{values\_to =} \StringTok{"value"}\NormalTok{) }\SpecialCharTok{\%\textgreater{}\%}
  \FunctionTok{group\_by}\NormalTok{(variable, value, sufficient\_sleep) }\SpecialCharTok{\%\textgreater{}\%}
  \FunctionTok{summarise}\NormalTok{(}\AttributeTok{count =} \FunctionTok{n}\NormalTok{()) }\SpecialCharTok{\%\textgreater{}\%}
  \FunctionTok{ggplot}\NormalTok{() }\SpecialCharTok{+}
  \FunctionTok{geom\_bar}\NormalTok{(}
    \AttributeTok{mapping =} \FunctionTok{aes}\NormalTok{(}\AttributeTok{x =}\NormalTok{ value, }\AttributeTok{y =}\NormalTok{ count, }\AttributeTok{fill =}\NormalTok{ sufficient\_sleep),}
    \AttributeTok{position =} \StringTok{"identity"}\NormalTok{,}
    \AttributeTok{alpha =} \FloatTok{0.6}\NormalTok{,}
    \AttributeTok{stat =} \StringTok{"identity"}
\NormalTok{  ) }\SpecialCharTok{+}
  \FunctionTok{facet\_wrap}\NormalTok{(}\SpecialCharTok{\textasciitilde{}}\NormalTok{ variable, }\AttributeTok{scales =} \StringTok{"free"}\NormalTok{) }\SpecialCharTok{+}
  \FunctionTok{labs}\NormalTok{(}\AttributeTok{x =} \StringTok{"Value"}\NormalTok{, }\AttributeTok{y =} \StringTok{"Count"}\NormalTok{, }\AttributeTok{fill =} \StringTok{"Sufficient Sleep"}\NormalTok{)}
\end{Highlighting}
\end{Shaded}

\begin{verbatim}
## `summarise()` has grouped output by 'variable', 'value'. You can override using
## the `.groups` argument.
\end{verbatim}

\begin{center}\includegraphics[width=0.7\linewidth]{Chapter9_files/figure-latex/unnamed-chunk-3-1} \end{center}

\begin{Shaded}
\begin{Highlighting}[]
\NormalTok{sleep\_data }\SpecialCharTok{\%\textgreater{}\%}
  \FunctionTok{pivot\_longer}\NormalTok{(}\AttributeTok{cols =} \FunctionTok{c}\NormalTok{(Gender), }\AttributeTok{names\_to =} \StringTok{"variable"}\NormalTok{, }\AttributeTok{values\_to =} \StringTok{"value"}\NormalTok{) }\SpecialCharTok{\%\textgreater{}\%}
  \FunctionTok{group\_by}\NormalTok{(variable, value, sufficient\_sleep) }\SpecialCharTok{\%\textgreater{}\%}
  \FunctionTok{summarise}\NormalTok{(}\AttributeTok{count =} \FunctionTok{n}\NormalTok{()) }\SpecialCharTok{\%\textgreater{}\%}
  \FunctionTok{ggplot}\NormalTok{() }\SpecialCharTok{+}
  \FunctionTok{geom\_bar}\NormalTok{(}
    \AttributeTok{mapping =} \FunctionTok{aes}\NormalTok{(}\AttributeTok{x =}\NormalTok{ value, }\AttributeTok{y =}\NormalTok{ count, }\AttributeTok{fill =}\NormalTok{ sufficient\_sleep),}
    \AttributeTok{position =} \StringTok{"identity"}\NormalTok{,}
    \AttributeTok{alpha =} \FloatTok{0.6}\NormalTok{,}
    \AttributeTok{stat =} \StringTok{"identity"}
\NormalTok{  ) }\SpecialCharTok{+}
  \FunctionTok{facet\_wrap}\NormalTok{(}\SpecialCharTok{\textasciitilde{}}\NormalTok{ variable, }\AttributeTok{scales =} \StringTok{"free"}\NormalTok{) }\SpecialCharTok{+}
  \FunctionTok{labs}\NormalTok{(}\AttributeTok{x =} \StringTok{"Value"}\NormalTok{, }\AttributeTok{y =} \StringTok{"Count"}\NormalTok{, }\AttributeTok{fill =} \StringTok{"Sufficient Sleep"}\NormalTok{)}
\end{Highlighting}
\end{Shaded}

\begin{verbatim}
## `summarise()` has grouped output by 'variable', 'value'. You can override using
## the `.groups` argument.
\end{verbatim}

\begin{center}\includegraphics[width=0.7\linewidth]{Chapter9_files/figure-latex/unnamed-chunk-3-2} \end{center}

\begin{Shaded}
\begin{Highlighting}[]
\NormalTok{sleep\_data }\SpecialCharTok{\%\textgreater{}\%}
  \FunctionTok{pivot\_longer}\NormalTok{(}\AttributeTok{cols =} \FunctionTok{c}\NormalTok{(BMI), }\AttributeTok{names\_to =} \StringTok{"variable"}\NormalTok{, }\AttributeTok{values\_to =} \StringTok{"value"}\NormalTok{) }\SpecialCharTok{\%\textgreater{}\%}
  \FunctionTok{mutate}\NormalTok{(}\AttributeTok{value =} \FunctionTok{ifelse}\NormalTok{(value }\SpecialCharTok{==} \StringTok{"Normal"}\NormalTok{, }\StringTok{"Normal Weight"}\NormalTok{, value)) }\SpecialCharTok{\%\textgreater{}\%}
  \FunctionTok{group\_by}\NormalTok{(variable, value, sufficient\_sleep) }\SpecialCharTok{\%\textgreater{}\%}
  \FunctionTok{summarise}\NormalTok{(}\AttributeTok{count =} \FunctionTok{n}\NormalTok{()) }\SpecialCharTok{\%\textgreater{}\%}
  \FunctionTok{ggplot}\NormalTok{() }\SpecialCharTok{+}
  \FunctionTok{geom\_bar}\NormalTok{(}
    \AttributeTok{mapping =} \FunctionTok{aes}\NormalTok{(}\AttributeTok{x =}\NormalTok{ value, }\AttributeTok{y =}\NormalTok{ count, }\AttributeTok{fill =}\NormalTok{ sufficient\_sleep),}
    \AttributeTok{position =} \StringTok{"identity"}\NormalTok{,}
    \AttributeTok{alpha =} \FloatTok{0.6}\NormalTok{,}
    \AttributeTok{stat =} \StringTok{"identity"}
\NormalTok{  ) }\SpecialCharTok{+}
  \FunctionTok{facet\_wrap}\NormalTok{(}\SpecialCharTok{\textasciitilde{}}\NormalTok{ variable, }\AttributeTok{scales =} \StringTok{"free"}\NormalTok{) }\SpecialCharTok{+}
  \FunctionTok{labs}\NormalTok{(}\AttributeTok{x =} \StringTok{"Value"}\NormalTok{, }\AttributeTok{y =} \StringTok{"Count"}\NormalTok{, }\AttributeTok{fill =} \StringTok{"Sufficient Sleep"}\NormalTok{)}
\end{Highlighting}
\end{Shaded}

\begin{verbatim}
## `summarise()` has grouped output by 'variable', 'value'. You can override using
## the `.groups` argument.
\end{verbatim}

\begin{center}\includegraphics[width=0.7\linewidth]{Chapter9_files/figure-latex/unnamed-chunk-3-3} \end{center}

\begin{Shaded}
\begin{Highlighting}[]
\NormalTok{sleep\_data }\SpecialCharTok{\%\textgreater{}\%}
  \FunctionTok{pivot\_longer}\NormalTok{(}\AttributeTok{cols =} \FunctionTok{c}\NormalTok{(Physical), }\AttributeTok{names\_to =} \StringTok{"variable"}\NormalTok{, }\AttributeTok{values\_to =} \StringTok{"value"}\NormalTok{) }\SpecialCharTok{\%\textgreater{}\%}
  \FunctionTok{group\_by}\NormalTok{(variable, value, sufficient\_sleep) }\SpecialCharTok{\%\textgreater{}\%}
  \FunctionTok{summarise}\NormalTok{(}\AttributeTok{count =} \FunctionTok{n}\NormalTok{()) }\SpecialCharTok{\%\textgreater{}\%}
  \FunctionTok{ggplot}\NormalTok{() }\SpecialCharTok{+}
  \FunctionTok{geom\_bar}\NormalTok{(}
    \AttributeTok{mapping =} \FunctionTok{aes}\NormalTok{(}\AttributeTok{x =}\NormalTok{ value, }\AttributeTok{y =}\NormalTok{ count, }\AttributeTok{fill =}\NormalTok{ sufficient\_sleep),}
    \AttributeTok{position =} \StringTok{"identity"}\NormalTok{,}
    \AttributeTok{alpha =} \FloatTok{0.6}\NormalTok{,}
    \AttributeTok{stat =} \StringTok{"identity"}
\NormalTok{  ) }\SpecialCharTok{+}
  \FunctionTok{facet\_wrap}\NormalTok{(}\SpecialCharTok{\textasciitilde{}}\NormalTok{ variable, }\AttributeTok{scales =} \StringTok{"free"}\NormalTok{) }\SpecialCharTok{+}
  \FunctionTok{labs}\NormalTok{(}\AttributeTok{x =} \StringTok{"Value"}\NormalTok{, }\AttributeTok{y =} \StringTok{"Count"}\NormalTok{, }\AttributeTok{fill =} \StringTok{"Sufficient Sleep"}\NormalTok{)}
\end{Highlighting}
\end{Shaded}

\begin{verbatim}
## `summarise()` has grouped output by 'variable', 'value'. You can override using
## the `.groups` argument.
\end{verbatim}

\begin{center}\includegraphics[width=0.7\linewidth]{Chapter9_files/figure-latex/unnamed-chunk-3-4} \end{center}

\begin{Shaded}
\begin{Highlighting}[]
\NormalTok{sleep\_data }\SpecialCharTok{\%\textgreater{}\%}
  \FunctionTok{pivot\_longer}\NormalTok{(}\AttributeTok{cols =} \FunctionTok{c}\NormalTok{( Stress, BPressure, HRate, DSteps), }\AttributeTok{names\_to =} \StringTok{"variable"}\NormalTok{, }\AttributeTok{values\_to =} \StringTok{"value"}\NormalTok{) }\SpecialCharTok{\%\textgreater{}\%}
  \FunctionTok{group\_by}\NormalTok{(variable, value, sufficient\_sleep) }\SpecialCharTok{\%\textgreater{}\%}
  \FunctionTok{summarise}\NormalTok{(}\AttributeTok{count =} \FunctionTok{n}\NormalTok{()) }\SpecialCharTok{\%\textgreater{}\%}
  \FunctionTok{ggplot}\NormalTok{() }\SpecialCharTok{+}
  \FunctionTok{geom\_bar}\NormalTok{(}
    \AttributeTok{mapping =} \FunctionTok{aes}\NormalTok{(}\AttributeTok{x =}\NormalTok{ value, }\AttributeTok{y =}\NormalTok{ count, }\AttributeTok{fill =}\NormalTok{ sufficient\_sleep),}
    \AttributeTok{position =} \StringTok{"identity"}\NormalTok{,}
    \AttributeTok{alpha =} \FloatTok{0.6}\NormalTok{,}
    \AttributeTok{stat =} \StringTok{"identity"}
\NormalTok{  ) }\SpecialCharTok{+}
  \FunctionTok{facet\_wrap}\NormalTok{(}\SpecialCharTok{\textasciitilde{}}\NormalTok{ variable, }\AttributeTok{scales =} \StringTok{"free"}\NormalTok{) }\SpecialCharTok{+}
  \FunctionTok{labs}\NormalTok{(}\AttributeTok{x =} \StringTok{"Value"}\NormalTok{, }\AttributeTok{y =} \StringTok{"Count"}\NormalTok{, }\AttributeTok{fill =} \StringTok{"Sufficient Sleep"}\NormalTok{)}
\end{Highlighting}
\end{Shaded}

\begin{verbatim}
## `summarise()` has grouped output by 'variable', 'value'. You can override using
## the `.groups` argument.
\end{verbatim}

\begin{center}\includegraphics[width=0.7\linewidth]{Chapter9_files/figure-latex/unnamed-chunk-3-5} \end{center}

\begin{Shaded}
\begin{Highlighting}[]
\CommentTok{\# Data Split (Training \& Testing)}
\FunctionTok{set.seed}\NormalTok{(}\DecValTok{123}\NormalTok{)}
\NormalTok{train\_indices }\OtherTok{\textless{}{-}} \FunctionTok{sample}\NormalTok{(}\DecValTok{1}\SpecialCharTok{:}\FunctionTok{nrow}\NormalTok{(sleep\_data), }\FloatTok{0.7} \SpecialCharTok{*} \FunctionTok{nrow}\NormalTok{(sleep\_data)) }
\NormalTok{trainingSet }\OtherTok{\textless{}{-}}\NormalTok{ sleep\_data[train\_indices, }\DecValTok{1}\SpecialCharTok{:}\DecValTok{12}\NormalTok{]}
\NormalTok{testSet }\OtherTok{\textless{}{-}}\NormalTok{ sleep\_data[}\SpecialCharTok{{-}}\NormalTok{train\_indices, }\DecValTok{1}\SpecialCharTok{:} \DecValTok{12}\NormalTok{]}
\NormalTok{training\_Outcomes }\OtherTok{\textless{}{-}}\NormalTok{ sleep\_data[train\_indices, }\StringTok{"sufficient\_sleep"}\NormalTok{]}
\NormalTok{test\_Outcomes }\OtherTok{\textless{}{-}}\NormalTok{ sleep\_data[}\SpecialCharTok{{-}}\NormalTok{train\_indices, }\StringTok{"sufficient\_sleep"}\NormalTok{]}


\CommentTok{\#model \textless{}{-} glm(sufficient\_sleep \textasciitilde{} Age + Gender + Occupation + Physical + DSteps + BMI, }
\CommentTok{\#             data = trainingSet, family = "binomial")}

\CommentTok{\#cv\_model \textless{}{-} train(sufficient\_sleep \textasciitilde{} Age + Gender + Occupation + Physical + DSteps + BMI, }
\CommentTok{\#                  data = trainingSet, method = "glm", trControl = trainControl(method = "cv", number = 5))}


\CommentTok{\#predictions \textless{}{-} predict(model, newdata = testSet, type = "response")}

\CommentTok{\#threshold \textless{}{-} 0.5  }
\CommentTok{\#predicted\_classes \textless{}{-} ifelse(predictions \textgreater{}= threshold, "Sufficient", "Insufficient")}
\CommentTok{\#actual\_classes \textless{}{-} test\_Outcomes}
\CommentTok{\#accuracy \textless{}{-} sum(predicted\_classes == actual\_classes) / length(actual\_classes)}
\CommentTok{\#print(paste("Accuracy:", accuracy))}

\CommentTok{\# Model 1 Prediction}
\CommentTok{\#model\_1\_preds \textless{}{-} testSet \%\textgreater{}\%}
\CommentTok{\#  add\_predictions(model, type = "response") \%\textgreater{}\%}
\CommentTok{\#  mutate(}
\CommentTok{\#    outcome = if\_else(pred \textgreater{} 0.5, "Sufficient", "Insufficient")}
\CommentTok{\#  )}
\end{Highlighting}
\end{Shaded}


\end{document}
